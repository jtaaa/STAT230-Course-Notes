\chapter{Counting Techniques}
% ------------------------------------------------- %
\section{Counting Arguments}
If we have a sample space, \sS, of some experiment that has a \textbf{uniform distribution} (all sample points are equally likely), then we can calculate the probability of a compound event $A$ as the number of sample points in $A$ divided by the total number of sample points.
\[
    P(A) = \frac{k}{n}
\]
where $k$ is the number of sample points in $A$ and $n$ is the total number of sample points in the sample space.
\subsection*{Addition Rule}
Consider we can perform process 1 in $p$ ways and process 2 in $q$ ways. Suppose we want to do process~1 \textbf{or} process~2 \textbf{but not both}, then there are $p + q$ ways to do so.
\begin{example}
Suppose a keyboard only has 26 letters and 20 special characters (!\%\#\$), there are 46 ways in which a typist may type a \textbf{single} character. (Process 1: typing a letter. Process 2: typing a special character).
\end{example}
\subsection*{Multiplication Rule}
Again, consider we can perform process~1 in $p$ ways and process~2 in $q$ ways. Suppose we want to do process~1 \textbf{and} process~2, then there are $p \times q$ ways to do so. This is because \textbf{for each way} of doing process~1 we can do process~2 in $q$ ways.
\begin{example}
Suppose the same typist with the same keyboard wants to type a single letter \textbf{and} a single special character. The typist can do so in 520 ways, since there are 26 ways to select the letter and \textbf{for each} possible letter selection there are 20 possible special character selections.
\end{example}
\begin{info}
Try to associate \textbf{OR} and \textbf{AND} with \textbf{addition} and \textbf{multiplication} respectively in your mind.
\end{info}
\begin{info}
 Often times, \textbf{OR}'s and \textbf{AND}'s are not explicit or obvious so you must re-word your problem to identify implicit \textbf{OR}'s and \textbf{AND}'s.
\end{info}
\begin{example}
A young boy gets to pick 2 toys from a store for his birthday. How many ways can he pick 2 toys if the store contains 12 toys? He may pick the same toy multiple times and picks the toys at random.
\tcblower
We can re-word this problem as follows: A young boy selects one of 12 toys \textbf{and} again, selects one of 12 toys. Thus there are $12 \times 12 = 144$ ways in which he can select 2 toys. Furthermore, we have that since selections are random, each selection is equally likely. So the probability that the boy selects any pair of toys is 1/144.
\end{example}
\begin{info}
In this case the boy was allowed to select the same toy more than once. This is often referred to as \textbf{with replacement}. The addition and multiplication rules are generally sufficient to find probability of processes with replacement but if processes occur without replacement solutions become more complex and other techniques are often used.
\end{info}
\begin{info}
The phrase \textbf{at random} or \textbf{uniformly}, indicates that each point in the sample space is equally likely.
\end{info}
\begin{example}
Consider a farmer with 500 different seeds. How many ways can he select 3 seeds randomly to plant?
\tcblower
We can re-word this problem to become: A farmer selects one seed from 500 \textbf{and} then selects one seed of 499 \textbf{and} then one seed of 498. So there are $500 \times 499 \times 498$ ways to do so.
\par\bigskip
Now, how many ways can he select 5 and 50 seeds randomly?
\par\medskip
He can select 5 seeds in $500 \times 499 \times 498 \times 497 \times 496$ ways and 50 seeds in $500 \times \cdots \times 451$ ways.
\end{example}
\begin{info}
Generally, if there are $n$~ways of doing a process and it is done $k$~times \textbf{without replacement}, that is you can only do the process a specific way once, there are $n \times \cdots \times (n-k+1)$ ways to do it.
\end{info}
% ------------------------------------------------- %
\section{Counting Arrangements}
When the sample space of a process is a set of arrangements of elements, like~$\{\,abc, acb, bac, bca, cab, cba\,\}$, the sample points are called the \textbf{permutations}. Assuming all $n$ elements we are arranging are unique, how many sample points are there?
\smallskip\par
Consider trying to fill $n$ boxes: \pboxes{0}\,.
We have $n$ ways to fill the first box (each element can go in the first box), \textbf{and} we have $(n-1)$ ways to fill the second box, \textbf{and} so on until we have 1 way to fill the n\ts{th}~box. So there are $n \times (n-1) \times \cdots \times 1$ total permutations in the sample space.
\begin{example}
Consider the letters of the word ``fiesta''. A baby (who cannot spell) randomly rearranges the letters of the word. What is the probability that ``fiesta'' is the outcome?
\tcblower
There are six boxes to fill: \pboxes{6}\,.
We have 6 ways to fill the first position, 5 ways to fill the second and so on until we have 1 way to fill the 6\ts{th} position. The number of points in the sample space is $6 \times 5 \times 4 \times 3 \times 2 \times 1 = 720$. So the probability of each outcome in the sample space is 1/720.
\end{example}
\begin{example}
Consider the letters of the word ``snake''. 
If arranged randomly what is the probability that the word formed begins with a vowel?
\tcblower
There are five boxes to fill: \pboxes{5}\,. There are two ways to fill the first box: 
\begin{center}
\pbox{a}\pboxes{4}\ and \pbox{e}\pboxes{4}
\end{center}
\textbf{and} for each of these ways there are four remaining boxes to fill. The number of ways to fill the 4 remaining boxes is $4 \times 3 \times 2 \times 1 = 24$ so the total number of outcomes in which the first letter is a vowel is $2 \times 24 = 48$. Therefore, the probability of the event occurring is $\frac{48}{\text{number of sample points}}$.
\smallskip\par
The five boxes can be filled by any letter to obtain a point in the sample space, so there are $5 \times 4 \times 3 \times 2 \times 1 = 120$ sample points. So the probability of the event occurring is $48/120 = 4/15$.
\end{example}
\begin{example}
Suppose we have 7 meals to distribute randomly to 7 people (one each). Three of the meals are gluten free and the other four are not. Of the 7 people, two of them cannot eat gluten. How many ways are there to distribute the meals without giving gluten to someone who cannot eat it?
\tcblower
We can liken this to the boxes example with each person being a box. Let the first two boxes be the people who cannot eat gluten. We have
\begin{center}
\pboxes{7}
\end{center}
Since we cannot place a gluten meal in boxes~1 or~2, we have that we have 3 ways to fill box~1 then 2 ways to fill box~2. So there are 6 ways distribute meals to the gluten-free people. We have
\begin{center}
\pbox{G}\pbox{G}
\pboxes{5}
\end{center}
Now there are 5 boxes to be filled with any of 5 meals. So there are $5 \times 4 \times 3 \times 2 \times 1 = 120$ ways to distribute the meals to the other 5 people. This is an implicit \textbf{and} statement, thus there are $6 \times 120 = 720$ ways to distribute the meals.
\end{example}
% ------------------------------------------------- %
\section{Notations}
Because some calculations occur very frequently in statistics we define a notation that helps us to deal with such problems.
\subsection*{Factorial}
We define $n!$ for any natural number $n$ to be
\[
    n! = n \times (n-1) \times (n-2) \times \cdots \times 1
\]
and in order to maintain mathematical consistency we define $0!$ to be 1. This is the number of arrangements of $n$ possible unique elements, using each once.
\subsection*{$\mathbf{n}$ to $\mathbf{k}$ Factors}
We define $\tofact{n}{k}$ to be
\[
    \tofact{n}{k} = n \times (n-1) \times \cdots \times (n-k+1) = \frac{n!}{(n-k)!}
\]
This is the number of arrangements of length $k$ using each element, of $n$ possible unique elements, at most once.
\subsection*{Power of}
As in ordinary mathematics $n^k = \underbrace{n \times n \times \cdots \times n}_k$. This represents the number of arrangements that can be made of length $k$ using each element, of $n$ possible unique elements, as often as we wish (with replacement).
\begin{info}
For many problems it is simply impractical to try to count the number of cases by conventional means because of how big the numbers become. Notations such as $n!$ and $n^k$ allow us to deal with these large numbers effectively.
\end{info}
\begin{example}
An evil advertising company randomly chooses 7-digit phone numbers to call to try to sell products. Find the probabilities of the following events:
\begin{itemize}[noitemsep, topsep=4pt plus 2pt minus 1pt]
    \item $A$: the number is your phone number
    \item $B$: the first three number are less than 5
    \item $C$: the first and last numbers match your phone number
\end{itemize}
Now assume that all 7-digits are unique (chosen without replacement):
\begin{itemize}[noitemsep, topsep=4pt plus 2pt minus 1pt]
    \item $D$: the number is 210-3869
    \item $E$: the first three number are less than 5
    \item $F$: the first and last numbers are 1 and 2 respectively
\end{itemize}
\tcblower
$A$:\hspace{5mm}The initial sample space contains all the ways that one can select 7 numbers from the numbers~0 to~9 \textbf{with replacement}. There are 10 choices for each of the seven numbers, therefore the sample space contains $10^7$ points. Thus, since all points are equally likely, $P(A) = 1/10^7$.
\par\smallskip
$B$:\hspace{5mm}Now if the first three numbers are less than 5, there are 5 ways (0 to 4) to select each of the first three numbers and there are 10 ways to select each of the next four numbers. So there are $5^3 \times 10^4$ points in $B$. Therefore, $P(B) = \frac{5^3 \times 10^4}{10^7}$
\par\smallskip
$C$:\hspace{5mm}There is only one way to select the first number such that it matches your number and the same is true for the last number. Thus, we must only consider the middle digits. There are 10 choices each for the middle five numbers, so there are $10^5$ points in $C$. Therefore, $P(C) = 1/10^5$.
\par\medskip
$D$:\hspace{5mm}The new sample space contains all the ways that one can select 7 numbers from the numbers~0 to~9 \textbf{without replacement}. There are 10 choices for the first number, 9~for the second and so on until there are 4 choices for the last number. Thus, there are $\tofact{10}{7}$ points in the sample space and since each is equally likely, $P(D) = 1/\tofact{10}{7} = \frac{1}{10 \times 9 \times 8 \times 7 \times 6 \times 5 \times 4}$.
\par\smallskip
$E$:\hspace{5mm}If the first three numbers are less than five, there are 5 ways to select the first number, 4 for the second and 3 for the third, so there are $\tofact{5}{3}$ ways to select the first 3 numbers. The next 4 digits may be selected from any of the 7 digits that were not used as one of the first 3. So there are $7^{(4)}$ ways to select the final four digits. Therefore, there are $5^{(3)} \times \tofact{7}{4}$ points in $E$. So, $P(E) = \frac{\tofact{5}{3} \times \tofact{7}{4}}{\tofact{10}{7}}$.
\par\smallskip
$F$:\hspace{5mm}There is only one way to select the first and last digits as 1 and 2 respectively, so we must only consider the middle 5 digits. The 5 digits are selected from 8 numbers without replacement, so there are $8^{(5)}$ ways to do this. Therefore, $P(F) = \frac{\tofact{8}{5}}{\tofact{10}{7}}$.
\end{example}
% ------------------------------------------------- %
\section{Counting Subsets}
In many problems, you will encounter a sample space, \sS, of some experiment that consists of fixed-length subsets of some set.
\subsection*{Combinations}
We define $\binom{n}{k}$ to be the number of subsets of size $k$ that can be selected from a set of $n$ elements. We have
\[
    \binom{n}{k} = \frac{\tofact{n}{k}}{k!} = \frac{n!}{(n-k)!\,k!}
\]
It is read ``$n$ choose $k$''.
\begin{theory}{Derivation of Choose}
Suppose we have a set of~$n$ unique~elements and we wish to select a subset of size~$k$, such that $k \leq n$, and the elements of the subset are unique (selected without replacement). If we use the boxes metaphor we have $k$ empty boxes. 
\[
    \underbrace{\pboxes{0}}_k
\]
There are $n$ ways to select the first element of the subset, $(n-1)$ ways to select the second and so on until there are $(n-k+1)$ ways to select the k\ts{th} and last element.
\par\medskip
So there are $\tofact{n}{k}$ ways to fill the $k$ boxes \textbf{but} note that some of the subsets will contain all the same elements as each other but in varying order. These subsets are not unique since we do not care for the arrangement of the items in a subset. Each unique subset can be arranged to form $k!$ permutations of its $k$ elements. Thus, the number of unique subsets, $\binom{n}{k}$, multiplied by the number of arrangements of each subset, $k!$, is $\tofact{n}{k}$. Therefore, we have
\[
    \binom{n}{k} \times k! = \tofact{n}{k}
\]
So it follows that
\[
    \binom{n}{k} = \frac{\tofact{n}{k}}{k!}
\]
\end{theory}
% ------------------------------------------------- %
% Add tons of examples and stuff right about here
%   |   |   |   |   |   |   |   |   |   |   |   |
%   |   |   |   |   |   |   |   |   |   |   |   |
%   V   V   V   V   V   V   V   V   V   V   V   V
% (Give or take a few pt's) :0 #;target
% ------------------------------------------------- %
% ------------------------------------------------- %
\section{Properties of Combinations}
Here are a few properties of $\binom{n}{k}$.
\begin{itemize}
    \item $\tofact{n}{k} = \frac{n!}{(n-k)!} = n(n-1)^{k-1}$ for $k \geq 1$
    \item $\binom{n}{k} = \frac{n!}{(n-k)!\,k!} = \frac{\tofact{n}{k}}{k!}$
    \item $\binom{n}{k} = \binom{n}{n-k}$ for all $0 \leq k \leq n$
    \item $\binom{n}{0} = \binom{n}{n} = 1$
    \item $(1 + x)^n = \binom{n}{0} + \binom{n}{1}x + \binom{n}{2}x^2 + \cdots + \binom{n}{n}x^n$ \hfill (Binomial Theorem)
\end{itemize}
% ------------------------------------------------- %
\section{Counting Arrangements of Set with Repeated Elements}
Thus far we have only discussed counting arrangements of unique items. Now, we consider a case in which we want to count the number of unique arrangements of size $k$ of a set of $n$ elements that are not necessarily unique.
\par\smallskip
% Consider we have a set of $n$~elements of $k$~unique types, with $n_i$~elements of type~$i$, for~$i~=~1,2,\ldots,k$, such that $n_1 + \cdots + n_k = n$. The number of different ways of selecting an arrangement of size $n$ that uses all symbols is:
Consider we have a set of $n$~elements with $k$ of those elements being unique. Let $n_i$ be the number of appearances of the $i$\ts{th} element of the $k$ unique elements. Thus, $n_1 + \cdots + n_k = n$. The number of different ways of selecting an arrangement of size $n$ that uses all symbols is:
\[
    \binom{n}{n_1} \times \binom{n - n_1}{n_2} \times \binom{n - n_1 - n_2}{n_3} \times \cdots \times \binom{n_k}{n_k} = \frac{n!}{n_1!\,n_2!\,n_3!\cdots n_k!}
\]
\begin{theory}{Derivation}
Suppose we have a set of $n$~elements with $k$~being unique. Let the $k$ unique items be labelled $u_1$~to~$u_k$ and let $n_i$ be the number of appearance of $u_i$ in the set of $n$ elements. We want to form an arrangement of length $n$ so using the boxes metaphor we have
\[\underbrace{\pboxes{0}}_n\]
We must use each of the $n$ elements once so we must select $n_1$ boxes to fill with $u_1$'s. This can be done in $\binom{n}{n_1}$ ways. Next, we must select $n_2$ of the remaining $n-n_1$ boxes to fill with $u_2$'s, $n_3$ of the remaining $n-n_1-n_2$ boxes to fill with $u_3$'s, and so on until we must select $n_k$ of the $n-n_1-n_2-\cdots-n_{k+1} = n_k$ remaining boxes to fill with $u_k$'s. Therefore, there are
\[
    \binom{n}{n_1} \times \binom{n - n_1}{n_2} \times \cdots \times \binom{n_k}{n_k} =
    \frac{n!}{\cancel{(n-n_1)!}\,n_1!} \times \frac{\cancel{(n-n_1)!}}{(n-n_1-n_2)!\,n_2!} \times \cdots \times \frac{(n_{k-1} + n_k)!}{\cancel{n_k!}\,n_{k-1}!} \times \frac{\cancel{n_k!}}{0!\,n_k!}
\]
ways, which simplifies to,
\[
    \frac{n!}{n_1!\,n_2!\,n_3!\cdots n_k!}
\]
to do so.
% ------------------------------------------------- %
% Add tons of examples and stuff right about here
%   |   |   |   |   |   |   |   |   |   |   |   |
%   |   |   |   |   |   |   |   |   |   |   |   |
%   V   V   V   V   V   V   V   V   V   V   V   V
% (Give or take a few pt's) :0 #;target
% ------------------------------------------------- %
\end{theory}
