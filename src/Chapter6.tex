\chapter{Useful Sums and Series}
This chapter includes a few useful sums and series that show up in the following chapters.
\section{Geometric Series}
\[
    \sum_{i = 0}^{n-1} r^i = 1 + r + r^2 + \cdots + r^{n-1} = \frac{1-r^n}{1-r}
\]
For $\abs{r} < 1$, we have
\[
    \sum_{i = 0}^{\infty} r^i = 1 + r + r^2 + \cdots = \frac{1}{1-r}
\]
\begin{info}
Other identities can be obtained from this one by differentiation. For example we have
\[
    \deriv r\sum_{i = 0}^{\infty} r^i = \sum_{i = 0}^{\infty} ir^{i-1} = \deriv r\,\frac{1}{1-r} = \frac{1}{(1-r)^2}
\]
\end{info}
\section{Binomial Theorem}
The binomial theorem describes the algebraic expansion of powers of a polynomial.
\[
    (1 + t)^n = 1 + \binom{n}{1}t^1 + \binom{n}{2}t^2 + \cdots + \binom{n}{n}t^n = \sum_{x = 0}^{n} \binom{n}{x} t^x
\]
for any positive integer $n$ and real number $t$.
\par\smallskip
A more general form of this theorem that holds even when $n$ is not a positive~integer is
\[
    (1 + t)^n = \sum_{x = 0}^{\infty} \binom{n}{x} t^x,\:\for \abs{t} < 1
\]
\begin{info}
It is an important skill to be able to recognize if an infinite, or otherwise, polynomial with binomial coefficients can be reduced to a simple polynomial raised to a power.
\end{info}
\section{Multinomial Theorem}
The multinomial theorem is a generalization of the binomial theorem. It describes the algebraic expansion of powers of a sum in terms of powers of the terms in the sum.
\[
    (x_1 + x_2 + \cdots + x_m)^n = \sum_{k_1 + k_2 + \cdots + k_m = n} \binom{n}{k_1,k_2,\ldots,k_m} \prod_{t = 1}^{m} x_t^{k_t}
\]
Another common form in which this theorem may be represented is
\[
    (x_1 + x_2 + \cdots + x_m)^n 
    = \sum_{k_1 + k_2 + \cdots + k_m = n} \frac{1}{k_1!\,k_2!\cdots k_m!}(x_1^{k_1}x_2^{k_2}\cdots x_m^{k_m})
\]
\begin{info}
The summation is over all non-negative integers, $k_1,k_2,\ldots,k_m$ such that $k_1 + k_2 + \cdots + k_m = n$
\end{info}