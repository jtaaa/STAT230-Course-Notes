\chapter{Discrete Random Variables and Probability Functions}
\section{Random Variables}
A random variable (\rv~for short) is a numerical valued variable that represents the outcome of an experiment or process. Every random variable has a range associated with it, which is the set of all possible values the \rv~can take. We denote random variables by capital letters, e.g., $A$, $X$, $Z$.
\begin{example}
Suppose an experiment consists of tossing a coin three times. Let the random variable $X$ be the number of heads that are rolled. And let the random variable $Y$ be the number of tails rolled. Now, we have a nice short hand in that $X = 2$ is equivalent to the statement ``two heads were rolled''.
\par\smallskip
Moreover, we have useful equalities such as $X + Y = 3$ and $X = 3 - Y$. 

The ranges of $X$ and $Y$ are both $\{\, 0,1,2,3 \,\}$.
\end{example}
\begin{info}
It is very important to understand the purpose of \rv's since the remainder of this course features them heavily.
\end{info}
The formal definition of a random variable is a function that assigns a real number to each point in a sample space.
\begin{example}
Consider the same experiment as above. The sample space is
\[
    \{\, \text{HHH, HHT, HTH, HTT, THH, THT, TTH, TTT} \,\}
\]
Let us define $X$ the number of heads that are rolled and the sample point $a = \text{HHT}$. The value of the function $X(a) = 2$ it is found by counting the number of heads in $a$. The range of $X$ is $\{\, 0,1,2,3 \,\}$. Each of the outcomes $X=x$ represents an event, simple or compound. In this case they are:
\begin{center}
\begin{tabular}{c|r}
    $X$     & Event \\
    \hline
    0       & $\{\, \text{TTT} \,\}$ \\
    1       & $\{\, \text{TTH, THT, HTT} \,\}$ \\
    2       & $\{\, \text{THH, HTH, HHT} \,\}$ \\
    3       & $\{\, \text{HHH} \,\}$ \\
\end{tabular}
\end{center}
\end{example}
Since some outcome in the range must always occur for a random variable for each event in the sample space, the events of a random variable are mutually exclusive subsets of the sample space such that their union is the total sample space. For \rv~$X$ and outcome $x$, we have $X = x$ represents some event and we are interested in calculating its probability. We denote the probability of $X = x$ by $P(X = x)$.
\begin{info}
Since the union of the events of values of a random variable is the total sample space, we have
\[
    \sum_{x \in \range{X}} P(X=x) = 1
\]
\end{info}
\subsection*{Discrete Random Variables}
Discrete random variables take integer values or, more generally, values in a countable set. Recall that a set is countable if its elements can be placed in a one-to-one correspondence with a subset of the positive integers.
\subsection*{Continuous Random Variables}
Continuous random variables take values in some interval of real numbers like $(0,1)$ or $(0,\infty)$ or $(\infty,\infty)$. You should be aware that there are infinitely numerical non-integer values that a \rv~with $\range{0,1}$ could take. The values are separated by infinitesimally small intervals.
\section{Probability Function}
The probability function of a random variable $X$ is a function that maps the value of $X$ to the probability of that value. The probability function is represented by 
\[
    f(x) = P(X = x)\for x\in\range{X}
\]
The set of pairs $\{\, (x,f(x)) \ssep x\in\range{X} \,\}$ is called the \textbf{probability distribution} of~$X$.

\subsection*{Properties of Probability Functions}
The following two properties hold for all probability functions.
\begin{itemize}
    \item $f(x) \geq 0 \for\all x\in\range{X}$
    \item $\sum\limits_{x\in\range{X}} f(x) = 1$
\end{itemize}
\section{Cumulative Distribution Function}
\lipsum[3]
