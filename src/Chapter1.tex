\chapter{Introduction}
\section{Defining Probability}
% ---------------------------------------------- %
% Classical Definition
\subsection*{The Classical Definition}
The probability of an event is
\[  
    \frac{\text{the number of ways the event may occur}}{\text{the total number of possible outcomes}}
\]
provided all outcomes are equally likely.
\begin{example}
The probability of a fair dice landing on 3 is 1/6 because there is one way in which the dice may land on 3 and 6 total possible outcomes of faces the dice may land on. The sample space of the experiment, $\sS$, is $\{1,2,3,4,5,6\}$ and the event occurs in only one of these six outcomes.
\end{example}
\begin{info}
The main limitation of this definition is that it demands that the outcomes of a sample space are equally likely. This is a problem since a definition of ``likelyhood'' (probability) is needed to include this postulate in a definition of probability itself.
\end{info}
% ---------------------------------------------- %
% Relative Frequency Definition
\subsection*{The Relative Frequency Definition}
The probability of an event is the limiting proportion of times that an event occurs in a large number of repetitions of an experiment.
\begin{example}
The probability of a fair dice landing on 3 is 1/6 because after a very large series of repetitions (ideally infinite) of rolling the dice, the fraction of times the face with 3 is rolled tends to 1/6.
\end{example}
\begin{info}
The main limitation of this definition is that we can never repeat a process indefinitely so we can never truly know the probability of an event from this definition. Additionally, in some cases we cannot even obtain a long series of repetitions of processes to produce an estimate due to restrictions on cost, time, etc.
\end{info}
% ---------------------------------------------- %
% Subjective Definition
\subsection*{The Subjective Definition}
The probability of an event occurring is a measure of how sure the person making the statement is that the event will occur.
\begin{example}
The probability that a football team will win their next match can be predicted by experts who regard all the data of past matches and current situations to provide a subjective probability.
\end{example}
\begin{info}
This definition is irrational and leads to many people having different probabilities for the same events, with no clear ``right'' answer. Thus, by this definition, probability is not an objective science.
\end{info}

% ---------------------------------------------- %
% Probability Model
\subsection*{Probability Model}
To avoid many of the limitation of the definitions of probability, we can instead treat probability as a mathematical system defined by a set of axioms. Thus, we can ignore the numerical values of probabilities until we consider a specific application. The model is defined as follows
\begin{itemize}
    \item A sample space of all possible outcomes of a random experiment is defined.
    \item A set of events, to which we may assign probabilities, is defined.
    \item A mechanism for assigning probabilities to events is specified.
\end{itemize}