\chapter{Mathematical Probability Models}
\section{Sample Spaces}
A sample space,~\sS, is a set of distinct outcomes for an experiment or process, with the property that in a single trial, one~and only~one of these outcomes occurs. The outcomes that make up a sample space are called sample points or simply points.
\begin{example}
The sample space for a roll of a six-sided die is
\[
    \{a_1,a_2,a_3,a_4,a_5,a_6\}\hspace{5mm}\text{where $a_i$ is the event the top face is $i$}
\]
More simply we could define the sample space as
\[
    \{1,2,3,4,5,6\}
\]
\end{example}
\begin{info}
Note that a sample space of a probability model for a process is not necessarily unique. Often times, however, we try to chose sample points that are the smallest possible or ``indivisible''.
\end{info}
\begin{example}
If we define $E$ to be the event that the top face of a six-sided~die is even when rolled and $O$ to be the event the top-face is odd, then the sample space,~\sS, can be defined as
\[
    \{E,O\}
\]
This is the same process as Example 2.1.1 (rolling a six-sided die), so since the sample spaces differ, clearly, sample spaces are not unique. Moreover, if we are interested in the event that a 3 is rolled, this sample space is not suitable since it groups the event in question with other events.
\end{example}
A sample space can be either \textbf{discrete} or \textbf{non-discrete}. If a sample space is discrete, it consists of a finite or countably infinite number ``simple~events''. A countably infinite set is one that can be put into a one-to-one correspondence with the set of real numbers. For example, $\left\{1,\frac{1}{2},\frac{1}{3},\frac{1}{4},\ldots\right\}$ is countably infinite whereas $\{x|x\in\sR\}$ is not.
\subsection*{Simple Events}
An event in a discrete sample space is a subset of the sample space, i.e., $A \subset \sS$. If the event is indivisible, so as to only contain one point, we call it a simple event, otherwise it is a compound event.
\begin{example}
A simple event for a roll of a six-sided die is $A = \{a_1\}$ where $a_i$ is the event the top face is i. A compound event is $E = \{a_2,a_4,a_6\}$.
\end{example}

\section{Assigning Probabilities}
Let $\sS = \{a_1,a_2,a_3,\ldots\}$ be a discrete sample space. We assign probabilities, $P(a_i),\for i = 1,2,3,\ldots$ to each sample point $a_i$ such that the following two conditions hold
\begin{itemize}
    \item $0 \leq P(a_i) \leq 1$
    \item $\sum\limits_{\all i} P(a_i) = 1$
\end{itemize}
The set of probabilities $\{P(a_i)|i = 1,2,3,\ldots\}$ is called a \textbf{probability distribution} on \sS.