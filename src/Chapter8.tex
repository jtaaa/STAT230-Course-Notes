\chapter{Discrete Distributions}
As we briefly mentioned in the previous chapter, \textbf{probability distributions} are the set of pairs~$(x,f(x))$ for all possible outcomes~$x$ of a random variable~$X$. Many probability distributions appear commonly on \rv's of similar ``real-life'' processes. In this chapter we define a few of these common distributions on discrete random variables, when they occur and how to use them to calculate probabilities.
\begin{info}
It is important to understand distributions early-on. Distributions, probability functions and cumulative distribution functions are defined on random variables \textbf{not} experiments/processes or sample spaces.
\end{info}

\section{Uniform Distribution}
Suppose $X$ can take a finite set of consecutive values with each of the values being equally likely. That is $\range{X} = \{\, a,a+1,a+2,\ldots,b \,\}$ with each of $a,a+1,a+2,\ldots,b$ being equally likely. Then $X$ has a discrete uniform distribution and we denote it \\
$X \dist \duni$
\[
    f(x) = P(X = x) = \left\{
    \begin{array}{c@{\hspace{1em}}l}
        \displaystyle\frac{1}{b - a + 1} & \for\all x\in\range{X} \\[1em]
         0 & \ow 
    \end{array}\right.
\]
\begin{theory}{Derivation of Probability Function}
The probability of each value of the \rv~is easy to calculate since they are all equal and must add up to 1. Therefore, $k \times P(X = a) = 1$ where $k$ is the number of possible values of $X$. The number of possible values of $X$ is $b - (a - 1) = b - a + 1$ since $\range{X}$ is between $a$ and $b$ inclusive.
\end{theory}
\begin{info}
Another way to define the probability of each value of a random variable with this sample space is 
\[
\frac{1}{\text{Number of possible values in $\range{X}$}}
\]
\end{info}
\section{Hypergeometric Distribution}
Suppose we have a collections of $N$ objects which can be classified into two different types, successes and failures. There are $r$ successes and $N -r$ failures. We pick $n$ objects at random without replacement, and let the random variable $X$ be the number of successes obtained. $X$ has a hypergeometric distribution and we denote it \\
$X\dist\hype$
\[
    f(x) = P(X = x) =
    \frac{\displaystyle\binom{r}{x}\binom{N-r}{n-x}}{\displaystyle\binom{N}{n}},\:\for x\leq\min(r,n)
\]
\begin{theory}{Derivation of Probability Function}
We will use the counting techniques we previously learnt to calculate the probability function. We note that there are $\binom{N}{n}$ ways to select $n$ objects from the total of $N$ so the sample space contains $\binom{N}{n}$~points. Now the number of ways of choosing $x$ successes from the total of $r$ is $\binom{r}{x}$ \textbf{and independently} the number of ways to choose the remainder of objects, $n-x$, from the total remaining objects, $N - r$, is $\binom{N - r}{n - x}$. Thus the probability of~$X = x$ by the multiplication rule is the product of those expressions divided by the number of points in the sample space, $\frac{\binom{r}{x}\binom{N-r}{n-x}}{\binom{N}{n}}$.
\end{theory}
\begin{info}
It is important to understand that the terms ``successes'' and ``failures'' are simply placeholder that represent a type of outcome and its complement. They could be replaced by ``wins'' and ``losses'', ``whites'' and ``colors'', or any other titles that are distinct groups with a union that spans the whole sample space.
\end{info}
\begin{info}
    This is used when we know how many items (n) are chosen at random from a set with two different types and we know the amount of each type in the set.
\end{info}
\begin{example}
    There is a basket with 11 fruit, 9 apples and 2 oranges. 4 fruit are picked at random from the basket. Let random variable $X$ be the number of apples selected. Find $f(x)=P(X=x)$. Then find $f(3)$. \\
    $X\dist\hype$. $N=11,n=4,r=9$.
    \[f(x)=P(X=x)=\displaystyle\frac{\displaystyle\binom{9}{x}\binom{2}{4-x}}{\displaystyle\binom{11}{4}}\for x\leq4\]
    Hence \[f(7)=P(X=7)=\displaystyle\frac{\displaystyle\binom{9}{3}\binom{2}{1}}{\displaystyle\binom{11}{4}}\approx0.509\]
\end{example}
\begin{example}
    15 cards are drawn from a deck of 52 at random. Let $X$ be the number of red cards drawn. Find $f(x)=P(X=x)$. Then find $f(7)$. \\
    $X\dist\hype$. $N=52,n=15,r=26$.
    \[f(x)=P(X=x)=\displaystyle\frac{\displaystyle\binom{26}{x}\binom{26}{15-x}}{\displaystyle\binom{52}{15}}\for x\leq15\]
    Hence \[f(7)=P(X=7)=\displaystyle\frac{\displaystyle\binom{26}{7}\binom{26}{8}}{\displaystyle\binom{52}{15}}\approx0.229\]
\end{example}