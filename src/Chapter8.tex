\chapter{Discrete Distributions}
As we briefly mentioned in the previous chapter, \textbf{probability distributions} are the set of pairs~$(x,f(x))$ for all possible outcomes~$x$ of a random variable~$X$. Many probability distributions appear commonly on \rv's of similar ``real-life'' processes. In this chapter we define a few of these common distributions on discrete random variables, when they occur and how to use them to calculate probabilities.
\begin{info}
It is important to understand distributions early-on. Distributions, probability functions and cumulative distribution functions are defined on random variables \textbf{not} experiments/processes or sample spaces.
\end{info}

\section{Uniform Distribution}
Suppose $X$ can take a finite set of consecutive values with each of the values being equally likely. That is $\range{X} = \{\, a,a+1,a+2,\ldots,b \,\}$ with each of $a,a+1,a+2,\ldots,b$ being equally likely. Then $X$ has a discrete uniform distribution and we denote it \\
$X \dist \duni$
\[
    f(x) = P(X = x) = \left\{
    \begin{array}{c@{\hspace{1em}}l}
        \displaystyle\frac{1}{b - a + 1} & \for\all x\in\range{X} \\[1em]
         0 & \ow 
    \end{array}\right.
\]
\begin{theory}{Derivation of Probability Function}
The probability of each value of the \rv~is easy to calculate since they are all equal and must add up to 1. Therefore, $k \times P(X = a) = 1$ where $k$ is the number of possible values of $X$. The number of possible values of $X$ is $b - (a - 1) = b - a + 1$ since $\range{X}$ is between $a$ and $b$ inclusive.
\end{theory}
\begin{info}
Another way to define the probability of each value of a random variable with this sample space is 
\[
\frac{1}{\text{Number of possible values in $\range{X}$}}
\]
\end{info}
\section{Hypergeometric Distribution}
Suppose we have a collections of $N$ objects which can be classified into two different types, successes and failures. There are $r$ successes and $N -r$ failures. We pick $n$ objects at random without replacement, and let the random variable $X$ be the number of successes obtained. $X$ has a hypergeometric distribution and we denote it \\
$X\dist\hype$
\[
    f(x) = P(X = x) =
    \frac{\displaystyle\binom{r}{x} \binom{N-r}{n-x}}{\displaystyle\binom{N}{n}},\:\for x\leq\min(r,n)
\]
\begin{theory}{Derivation of Probability Function}
We will use the counting techniques we previously learnt to calculate the probability function. We note that there are $\binom{N}{n}$ ways to select $n$ objects from the total of $N$ so the sample space contains $\binom{N}{n}$~points. Now the number of ways of choosing $x$ successes from the total of $r$ is $\binom{r}{x}$ \textbf{and independently} the number of ways to choose the remainder of objects, $n-x$, from the total remaining objects, $N - r$, is $\binom{N - r}{n - x}$. Thus the probability of~$X = x$ by the multiplication rule is the product of those expressions divided by the number of points in the sample space, $\frac{\binom{r}{x}\binom{N-r}{n-x}}{\binom{N}{n}}$.
\end{theory}
\begin{info}
It is important to understand that the terms ``successes'' and ``failures'' are simply placeholder that represent a type of outcome and its complement. They could be replaced by ``wins'' and ``losses'', ``whites'' and ``colors'', or any other titles that are distinct groups with a union that spans the whole sample space.
\end{info}
\begin{info}
    This is used when we know how many items (n) are chosen at random from a set with two different types and we know the amount of each type in the set.
\end{info}
\begin{example}
There is a basket with 11 fruit, 9 apples and 2 oranges. 4 fruit are picked at random from the basket. Let random variable $X$ be the number of apples selected. Find $f(x)=P(X=x)$. Then find $f(3)$. \\
$X\dist\hype$. $N=11,n=4,r=9$.
\[
    f(x) = P(X = x) = \displaystyle\frac{\displaystyle\binom{9}{x} \binom{2}{4-x}} {\displaystyle\binom{11}{4}},\:\for x \leq 4
\]
Hence 
\[
    f(7) = P(X = 7) = \displaystyle\frac{\displaystyle\binom{9}{3} \binom{2}{1}} {\displaystyle\binom{11}{4}} \approx 0.509
\]
\end{example}
\begin{example}
15 cards are drawn from a deck of 52 at random. Let $X$ be the number of red cards drawn. Find $f(x)=P(X=x)$. Then find $f(7)$. \\
$X\dist\hype$. $N=52,n=15,r=26$.
\[
    f(x) = P(X = x) = \displaystyle\frac{\displaystyle\binom{26}{x} \binom{26}{15-x}} {\displaystyle\binom{52}{15}},\:\for x \leq 15
\]
Hence \[
    f(7) = P(X = 7) = \displaystyle\frac{\displaystyle\binom{26}{7} \binom{26}{8}} {\displaystyle\binom{52}{15}} \approx 0.229
\]
\end{example}
\section{Binomial Distribution}
Suppose we have an experiment with two distinct outcomes, success and failure, with the probability of a success being $p$ and a failure being $(1-p)$. The experiment is repeated $n$ times independently (these are called trials). Let the random variable $X$ be the number of successes obtained. $X$ has a binomial distribution. \\
$X\dist\bino(n,p)$
\[
    f(x) = P(X = x) = \displaystyle\binom{n}{x}p^x (1-p)^{n-x},\:\for x=1,\ldots,n
\]
\begin{theory}{Derivation of Probability Function}
Since there are $n$ positions in which to put the $x$ successes there are $\binom{n}{x}$ unique arrangements of successes and failures that satisfy ``$X = x$''. Each of these arrangements has probability $p^x (1-p)^{n-x}$ since the probability of obtaining $x$ successes is $p^x$ and the probability of obtaining $n-x$ failures is $(1-p)^{n-x}$. So the probability that $X = x$, that is that any of the arrangements occur, is the sum of the probability of each unique arrangement, $\binom{n}{x}p^x (1-p)^{n-x}$.
\end{theory}
\begin{info}
The above formula describes the probability of $x$ success and $(n-x)$ failures multiplied by the number of different ways of arranging those successes within the total number of trials of the experiment.
\end{info}
\begin{info}
Each of the n individual experiments is called a ``Bernoulli trial'' and the entire process of n trials is called a Bernoulli process or a Binomial process.
\end{info}
\begin{example}
A loaded coin is flipped 10 times, with a probability of a heads occurring being 0.4. Let random variable $X$ be the number of heads that occur. Find $f(x)=P(X=x)$, then find $f(3)$. \\
$X\dist\bino(10,0.4)$.
\[
    f(x) = P(X = x) = \displaystyle\binom{10}{x} (0.4)^x (0.6)^{10-x},\:\for x = 1,\ldots,10
\]
Hence 
\[
    f(3) = P(X = 3) = \displaystyle\binom{10}{3} (0.4)^3 (0.6)^7 \approx 0.215
\]
\end{example}
\begin{example}
A football season in a university league has 22 games. The probability of each game being abandoned (because of bad weather or other hazards) is 0.02. Let $X$ be the number of games abandoned throughout the whole season. Find $f(x)=P(X=x)$, then find $f(2)$ and $f(10)$. \\
$X\dist\bino(22,0.02)$.
\[
    f(x) = P(X = x) = \displaystyle\binom{22}{x} (0.02)^x (0.98)^{22-x},\:\for x = 1,\ldots,22
\]
Hence 
\[
    f(2) = P(X = 2) = \displaystyle\binom{22}{2} (0.02)^2 (0.98)^20 \approx 0.062
\]
and 
\[
    f(10) = P(X = 10) = \displaystyle\binom{22}{10} (0.02)^{10} (0.98)^{12} \approx \sn{5.196}{-12}
\]
\end{example}
\subsection{Comparison of Binomial and Hypergeometric Distributions}
The Binomial and Hypergeometric distributions are similar in that they both model the distribution of the number of successes in $n$ trials of an experiment. The difference is that the collection of objects in the hypergeometric distribution is selected from without replacement as apposed the Binomial distribution in which successes and failures do not affect the probability of future outcomes (with replacement).
\begin{info}
The Hypergeometric distribution is used when there is a fixed number of objects (successes and~failures) to choose from.
\par\smallskip
The Binomial distribution is used when there is no fixed number of objects to be selected from and instead we know the constant probability of a success for all the trials.
\end{info}
\begin{example}
Consider Lisa owns a car dealership and has only 150~red cars and 350~blue cars in stock. A rich~Swedish man enters and picks 100~cars randomly to purchase. Let $X$ be the number of red cars the Swede purchases.
\par\smallskip
Since we know the number of successes (150~red cars) and failures (350~blue cars) as well as the number of trials, we have that $X \dist \geom$ and 
\[
    f(x) = P(X = x) = \frac{\binom{150}{x} \binom{350}{100-x}}{\binom{500}{100}}
\]
Now, consider Lisa has run out of all her stock of cars. She goes to a Swedish car manufacturer's factory which is capable of producing any amount of cars. The factory has a 30\% chance of producing a red~car and otherwise produces a blue~car. Lisa orders 100~cars. Let $X$ be the number of red~cars she receives.
\par\smallskip
Since there is no fixed number of cars to choose from but we do know the probability of each car being a success, we have that $X \dist \bino(100, 0.3)$ and
\[
    f(x) = P(X = x) = \binom{100}{x} (0.3)^{x} (0.7)^{100-x}
\]
\end{example}